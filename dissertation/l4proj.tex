% REMEMBER: You must not plagiarise anything in your report. Be extremely careful.

\documentclass{l4proj}

    
%
% put any additional packages here
%
\usepackage{graphicx}
\usepackage[export]{adjustbox}
\begin{document}

%==============================================================================
%% METADATA
\title{Learning cryptography through gaming}
\author{Alexander Behrmann}
\date{April 6th, 2020}

\maketitle

%==============================================================================
%% ABSTRACT
\begin{abstract}
    Every abstract follows a similar pattern. Motivate; set aims; describe work; explain results.
    \vskip 0.5em
    ``XYZ is bad. This project investigated ABC to determine if it was better. 
    ABC used XXX and YYY to implement ZZZ. This is particularly interesting as XXX and YYY have
    never been used together. It was found that  
    ABC was 20\% better than XYZ, though it caused rabies in half of subjects.''
\end{abstract}

%==============================================================================

% EDUCATION REUSE CONSENT FORM
% If you consent to your project being shown to future students for educational purposes
% then insert your name and the date below to  sign the education use form that appears in the front of the document. 
% You must explicitly give consent if you wish to do so.
% If you sign, your project may be included in the Hall of Fame if it scores particularly highly.
%
% Please note that you are under no obligation to sign 
% this declaration, but doing so would help future students.
%
%\def\consentname {My Name} % your full name
%\def\consentdate {20 March 2018} % the date you agree
%
\educationalconsent
\def\consentname {Alexander Behrmann} % your full name
\def\consentdate {20 March 2020} % the date you agree

%==============================================================================
\tableofcontents

%==============================================================================
%% Notes on formatting
%==============================================================================
% The first page, abstract and table of contents are numbered using Roman numerals and are not
% included in the page count. 
%
% From now on pages are numbered
% using Arabic numerals. Therefore, immediately after the first call to \chapter we need the call
% \pagenumbering{arabic} and this should be called once only in the document. 
%
% Do not alter the bibliography style.
%
% The first Chapter should then be on page 1. You are allowed 40 pages for a 40 credit project and 30 pages for a 
% 20 credit report. This includes everything numbered in Arabic numerals (excluding front matter) up
% to but excluding the appendices and bibliography.
%
% You must not alter text size (it is currently 10pt) or alter margins or spacing.
%
%
%==================================================================================================================================
%
% IMPORTANT
% The chapter headings here are **suggestions**. You don't have to follow this model if
% it doesn't fit your project. Every project should have an introduction and conclusion,
% however. 
%
%==================================================================================================================================
\chapter{Introduction} \label{sec:Introduction}

% reset page numbering. Don't remove this!
\pagenumbering{arabic} 


\section{Motivation}

Nowadays, lectures tend to be overcrowded and there is a lack of interaction between students and lecturers \citep{lehmann_theory-driven_2014}.
Due to this fact, the material presented in lectures in result is not conveyed correctly.
Moreover, the students do not have the opportunity to check their understanding since interactions in large classroom are not feasible.
Hence, the learning quality tends to diminish which leaves both lectures and students unsatisfied.
As technology has become common in our daily lives, it can be used to resolve this issue.
Gamifications and serious games are used to educate its players in various subjects.
Therefore, they could be used to provide much needed feedback and enhance their knowledge in a particular subject.

Cyber Security is an important issue today, especially because of the growth of the Internet of Things (IoT) \citep{abomhara_cyber_2015}.
It is a priority to have a good understanding of how systems work and to able to identify relevant vulnerabilities that can exist in the world \citep{abomhara_cyber_2015}.
This understanding usually comes from people taking courses in cyber security.
However, as stated in the previous paragraph, it is difficult to test students' understanding in lectures.
Hence, some students might make some mistakes later on in their careers that might cause an attack 
which could have been avoided with better feedback.
Gamifications and serious games provide a framework which could solve this issue 
by giving feedback to the student concerning their comprehension of cyber security, by enhancing and complementing the material taught in lectures,
and by motivating them to learn more due to the fun nature of these games.

Finally, \citet{schneier_security_nodate} claims that cryptography is powerful when it is implemented correctly.
Hence, attackers target the implementation around cryptography and not the cryptography itself \citep{schneier_security_nodate} \citep{lazar_why_nodate}.
Since implementing cryptography into a system requires thorough understanding of how it works,
it is essential that students get appropriate feedback concerning how these algorithms work.
This is hindered by the lack of interaction (it is difficult for students to ask questions, for example) and feedback during lectures.
Moreover, after an extensive research of cyber security games presented in 2.3, there are very few games that teach people about cryptography.
Hence, creating a game about cryptography will not only educate students that take a cyber security course at university, 
but also computer scientists and people around the world who want to learn more about cryptographic implementations.

\section{Aim}

The aim of this project is to complement the Cyber Security Fundamentals (CSF) with an open-world game 
which covers material delivered in this course.
The CSF course is taught at the University of Glasgow during the course of ten weeks and covers the following cyber security topics:
Penetration Testing, Cryptography, Web Application Attacks, and Digital Forensics.
In 2019, a group of students pursuing a master's degree in cyber security developed an open-world game 
that would complement the CSF course. 
Unfortunately, the overall result had some issues in it which required redesigning.
These issues are presented in the third chapter of this paper.
The main aim of this project is to provide feedback and enhance students knowledge in cryptography
by building a quest. 
This quest covers and extends the material presented in the course about cryptography.
The material present in the course about cryptography is the following: cryptography fundamentals 
and the three types of cryptographic algorithms (symmetric, asymmetric, and hashing algorithms).
The game aims to give students practice in understanding and manipulating cryptographic algorithms.
Moreover, the game aims to teach students common mistakes made when implementing systems that use cryptography.
Finally, this also allows to build a game that covers cryptography in detail as there are very few games that have this theme.

A second goal of this project is to show that lectures can be made more interactive by the use of educational games.
As stated in the motivation section, the lack of interaction causes lack of feedback which in turn can cause a lack of understanding.
In a field like cyber security, it is essential that students get appropriate feedback to make good decisions later on.
Hence, this game could improve the learning experience of students and make it more enjoyable.

\section{Paper summary}

This chapter presented the motivation and the goals of this project.
What follows is an outline of the different sections of this paper:
\begin{itemize}
    \item Chapter 2 presents what gamifications and serious games are, delivers some cryptography basics, and reviews cyber security games.
    \item Chapter 3 analyses the issue at hand, and formulates requirements for this project.
    \item Chapter 4 provides the reasoning behind the design of the cryptography quest.
    \item Chapter 5 provides a detailed overview of the implementation of the quest.
    \item Chapter 6 demonstrates the planned evaluation and the expected results are explained.
    \item Chapter 7 gives an overview of the whole project and gives directions to future work.
\end{itemize}

%==================================================================================================================================
\chapter{Background}

This chapter is going to go over the research that has been completed in the area of gamification and serious games applied to cyber security and more precisely cryptography.
The main focus will be on the applications of cryptography and not the building of a cryptographic algorithm.
The first section of this chapter will discuss gamification and serious games. It will demonstrate why it is a possible system for education and prove why it is a workable approach.
Finally, an attempt at using gamification to educate people will be shown.
The next part of this chapter will discuss cryptography, analyse various applications of cryptography, and why there exists a need to educate users in a better usage.
As the game is designed for people who have no prior knowledge of cryptography, some cryptographic fundamentals will be discussed.
The third section of this chapter will explore some games that have been built in the domain of cybersecurity, and these games will be reviewed according to 
their gamification and cyber security content.
Finally, this section will be concluded by stating the reasons why there is a need to delve into more research in the domain of cryptography and educational games.

\section{Gamification and Serious Games}

\subsection{Gamification}
Gamification is the usage of game-like elements in a non-gaming situation \citep{aparicio_analysis_2012}.
\citet{aparicio_analysis_2012} claim that it used to attract people to tasks that are not normally considered as entertaining, like education for example.
The research will mostly focus on gamification applied to education as game-like elements are going to be applied to cryptography.
Some gamification characteristics will be identified and some examples will be demonstrated to show that gamification works.

According to \citet{hsiao_brief_2007}, motivation is important for learning as learning requires a considerable amount of effort. 
Games have the following traits: they are fun, they have a goal, they are interactive, they have outcomes and provide feedback, they have some form of challenge, 
they have a story, they require us to solve problems. Through these elements, games motivate and engage people \citep{hsiao_brief_2007}. 
However, it is mainly through problem solving that players will be learning as people will be improving their skills by playing and solving challenges \citep{hsiao_brief_2007}.
To apply gamification to a certain context, \citet{cugelman_gamification:_2013} has identified 7 important components for a good gamification. 
These include "goal-setting", "capacity to overcome challenges", "providing feedback on performance", "reinforcement", "progress comparison", "social connectivity", 
and "fun and playfulness" \citep{cugelman_gamification:_2013}. 
However, \citet{nicholson_recipe_2015} claims that there are other models to motivate players to learn. Instead of providing motivation from rewards, 
users can find their own reasons to play the game. This could come from their desire to improve their skills in cryptography, their interest in cryptography, or 
willing to advance the storyline of the character since they identify themselves to him for example.
Also, gamification allows to put concepts into practice, especially in the field of cyber security \citep{wolfenden_gamification_2019}. 
This is true since people apply learned concepts in a potentially real environment which improves knowledge retention. 
People struggle to apply concepts if they are not used in practice, and gamification allows to put these theoretically learned skills into practice.
In addition, gamification in cyber security allows to screen people,
and check fields where they potentially lack the necessary skills for a safe usage \citep{adams_cybersecurity_2015}.

\citet{akpolat_enhancing_2014} conducted a study about the gamification of extreme software engineering practices. These practices were pair programming, testing, and refactoring.
For six weeks, they studied 50 students taking a university course held over 14 weeks. During this course, they learned these software engineering practices,
and had the opportunity to apply these ideas to develop a software system. Students were split into five teams, each person had a specific role in the team,
and they would have an eight hour work day per week to work on their project. For six weeks, students engaged with this game. Each week, the teams competed for a "challenge cup", 
and this cup was awarded to the team who won the challenge for that week. At the end of the course, there would be a final winner (the team who won the most challenges),
and each team member was allowed to print something using a 3D printer. Each week, students were assigned a challenge, 
and each challenge covered one of the extreme software engineering practices. The teams were evaluated by a "facilitator", 
and students were not evaluated individually to avoid competition within teams.
Some gamification characteristics described by \citet{cugelman_gamification:_2013} can be found in this application:
\begin{itemize}
    \item "Goal-setting": Each week, team would apply a software engineering practice;
    \item "Capacity to overcome challenges": The challenges could be completed by applying the practice of the week;
    \item "Providing feedback on performance": The evaluators picked out a team member of each to evaluate their team's understanding of the concept, 
    hence providing feedback on how well the whole team understood the material presented that week;
    \item "Reinforcement": A team would win a cup every week for the best use of a software practice;
    \item "Progress comparison": Teams were competing against each other and there was an overall leaderboard for students to compare their progress;
    \item "Social connectivity": Since people were working in teams to complete a challenge, there was a lot of "social connectivity";
    \item "Fun and playfulness": Students were having a real day in industry every week which is a totally different environment than a monotonous lecture at the university. 
\end{itemize}
\citet{akpolat_enhancing_2014} found out that students engaged more with a practice when it was the practice of the week. Moreover, once that weekly challenge was over, 
students would still use that practice the following weeks. They also found out that when students were confronted a second time with a practice, 
students were using it in a much more intensive manner. Finally, \citet{akpolat_enhancing_2014} conducted two surveys, one in the middle of the game, and one at the end of the game.
They concluded that students thought that their learning through this game was a success if it was compared to another university course without this gamification concept.
Another conclusion was that students' opinion about their learning and coding performance was much better than it was before the challenge.

In conclusion, this section provided a definition of gamification, introduced some characteristics of gamification. 
At the end, a study of an application of gamification was described in a software engineering environment as a proof that gamification works.
The next section will cover serious games and its applications.

\subsection{Serious Games}

\citet{susi_serious_2007} define serious games as games that are used for "training, advertising, simulation, or education that are designed to run
on personal computers or video game consoles". The primary purpose of these games is not entertainment, but is more like a simulation of the real-world
to train people \citep{susi_serious_2007}. This section will present some characteristics of serious games, 
and an application of serious games to demonstrate that they provide some educational results.

\citet{blumberg_serious_2012} have identified some characteristics that define serious games. These characteristics are the following: "Immersion", "Identity",
"Interactivity", "Agency or Control", "Challenge", "Narrative" and "Feedback" \citep{blumberg_serious_2012}.
"Immersion" is about the people feeling that they are part of the world of the game while playing it. 
"Identity" refers to the player identifying themselves with their avatar or character they are playing in-game.
"Interactivity" is all about the feedback players get for their in-game decisions, and how these decisions affect the chain of events in the game.
"Agency or Control" defines how the player interacts with the game itself. This not only encompasses the control mechanisms, 
but also how gamers interact with the game which makes this point tightly linked with the previous "Interactivity" point.
"Challenge" reflects to the complexity of a task, but also to player's ability to accomplish that task. Therefore, there is a requirement of a balance between these two elements.
"Narrative" is required so that players have some attachment point to that game has more diversity rather than a single event.
Finally, "Feedback" gives the players some details about their actions, and give them a reason to continue playing the game.
Moreover, \citet{susi_serious_2007} claim that entertainment is also a characteristic of serious games that improves the final goal of that game.
They also claim that pedagogy is what differentiates serious games from a simple game. However, they say that pedagogy should come after entertainment, and not before.
The next paragraph will be a case study of a serious game as a proof of concept.

What follows is an application of the serious games methodology. \citet{panzoli_interaction_2017} present some games that aim to replicate real life interactions between doctors
in a simulated operating room environment. In their research paper, they featured a game called "3D Virtual Operating Room" \citep{panzoli_interaction_2017}.
Its main aim is to encourage communication and cooperation between the different actors in a surgery room.
The problem the game is solving, is the following: the main cause of failures in an operating room environment is due to miscommunication 
which can have dramatic consequences like the patient's death \citep{panzoli_interaction_2017}. 
The game models several rooms that are important during a surgery (pre-operating room, operating room, and post-operating), 
is a first person point of view game.
Throughout the gaming session, the players actions are recorded and they receive some feedback at the end of the session.
The game tries to replicate real-life scenarios where mistakes often occur, 
and its aim is to assess the reaction of the participants to the mistakes that might arise \citep{panzoli_interaction_2017}.
The game presents the following serious game characteristics identified by \citet{blumberg_serious_2012}:
\begin{itemize}
    \item "Immersion": Immersion is provided by the realistic environment, presence of surgical protocols, and the first person point of view;
    \item "Identity": Identity is provided via the first person point of view, and the multiplayer nature of this game;
    \item "Interactivity": There are a set of interactions the player can do, each of those will have their own set of consequences down the line, 
    the player is also interacting with other people in real-time to solve the challenge;
    \item "Agency or Control": The player can interact with the environment (drawers, for example), but also with other players
    \item "Challenge": The aim of this game is to avoid recreating the mistakes that are commonly made in a surgery;
    \item "Narrative": The patient being operated on provides some narrative (his disease, for example);
    \item "Feedback": At the end of the session, the players are given feedback concerning their in-game actions.
\end{itemize}
\citet{panzoli_interaction_2017} claim that "3D Virtual Operating Room" is a successful game due to its immersiveness, and its multiplayer nature.
Hence, they decided to implement their own game based on the successful design choices of that game.  

\subsection{Conclusion}

In conclusion, this section about Gamification and Serious Games presented a definition of these two concepts, provided their characteristics, 
and demonstrated an example of both concepts. Although both ideas seem to be similar, they are not. 
Both try to educate people through the use of technology and converting technical ideas into games.
However, gamification is created to target a large audience of people that do not especially have the required knowledge in the field by adding game-like elements 
to make the learning activity more exciting.
On the other hand, Serious Games are created to simulate the real world, and prepare the players of that game for a real world environment in which they are going to act.
The next section will present cryptography under its many forms, and explain why there is a need to educate people in this field.

\section{Cryptography}

Cryptography is a science of writing secret code to preserve confidentiality and integrity across time and space in the presence of an adversary \citep{kessler_overview_2016} 
\citep{savage_cse_2019}. Cryptography is used to protect data from being stolen or modified, and it is also used for user authentication \citep{kessler_overview_2016}. 
In this section, there will be a brief historic overview of early cryptographic algorithms; an overview of cryptographic paradigms and algorithms; 
an analysis of current cryptographic applications and the mistakes that are made in these applications.

\subsection{Origins of Cryptography}

\citet{kessler_overview_2016} claims that the first documented appearance of cryptography was as early as 1900 B.C. when an inscription was written with non-common hieroglyphs.
Some people claim that cryptography was invented just some time after the invention of writing because diplomatic messages and war orders had to be delivered securely 
\citep{kessler_overview_2016}. Nineteen centuries after the first documented use of cryptography, Julius Caesar used a substitution cipher to send his orders to his officers 
\citep{anderson_security_2008}. Substitution ciphers have been widely used across the centuries, however these ciphers are quite easy to break with a pen and pencil 
\citep{anderson_security_2008}. One way to make substitution ciphers harder to break is to use a stream cipher: the Vigenère takes two inputs, the plaintext and the key, and it
returns a cipher \citep{anderson_security_2008}. The positions of the letters in the alphabet are summed together, then the sum is divided by the number of letters in the alphabet, 
finally the remainder of this division defines the position of the letter in the alphabet that should be used in the cipher \citep{anderson_security_2008}. 
A vulnerability in this algorithm is that if the plaintext is longer than the key and if the cipher is long enough, the cipher will reveal repeated letter patterns 
that can be used to guess the key \citep{anderson_security_2008}. To make the stream cipher more secure, a plaintext and a key of equal length are used. 
This is the basis for the One-Time Pad. The One-Time Pad works by taking a binary plaintext and a binary key and XORing them together \citep{savage_cse_2019}.
This algorithm works as long as each key is used exactly once, all keys are equally likely, the plaintext and the key are of the same length, and there are as many plaintexts 
as there are keys \citep{savage_cse_2019} \citep{anderson_security_2008}. This system provides perfect secrecy, however it does not provide integrity nor authenticity 
\citep{savage_cse_2019}. The name One-Time Pad comes from World War 1 where an agent would have the key printed out on some material, and each time a key was used, the key would
be ripped off and burnt \citep{anderson_security_2008}. 

Another way to make substitution cipher harder to break is to use a block cipher: the Playfair system works by placing the
alphabet into a five by five grid without the letter j, and shifting it by the key. The plaintext is then modified by replacing "j" by "i", 
splitting double letters with an "x" in between, and finally splitting the text into letter pairs and adding a "z" at the end if the final letter has no pair \citep{anderson_security_2008}.
Finally, to encrypt the letter pairs, two techniques are used: if the letters are on the same row, the letters in the cipher will be the immediate next letters in the grid 
to the letters in the plaintext; otherwise the letters from the plaintext are situated at two corners of a grid, and they are replaced by the letters that are located in the corner to the left or right of them. 
What follows is an example of this algorithm. Suppose the key is "crypto", and the plaintext is "just shifting letters".
The key is placed in a grid represented in Table \ref{tab:key}. Then, by applying the rules to plaintext, "iu" "st" "sh" "if" "ti" "ng" "le" "tx" "te" "rs" "is" obtained.
Finally, by using the table, the cipher "fx" "ze" "nk" "kg" "pk" "mh" "so" "pz" "ek" "tm" is obtained. 
An issue with this algorithm is that a single letter is changed in the plaintext,
then a single letter will change in the cipher, 
this allows the grid to be figured out if enough cipher text is given or some probable words are given \citep{anderson_security_2008}.
A block cipher can become much more secure if a larger block length is used (in the example above, the block size is two), 
this makes an important foundation for modern block cipher algorithms
\citep{anderson_security_2008}. This example demonstrates the Electronic Code Book (ECB) mode of encryption: the key encrypts a block of plaintext to make a block of cipher \citep{kessler_overview_2016}.
This mode of encryption is the most common, however if two plaintext blocks are the same and are encrypted using this method, the cipher block will be the same, 
and this makes this mode of encryption vulnerable \citep{kessler_overview_2016}. Another mode that uses the concept of block ciphers is called Cipher Block Chaining (CBC).
Cipher Block Chaining works by xORing the previous cipher block with the current plaintext block, and by using afterwards a block cipher encryption algorithm
on the result of that operation to create a new cipher block \citep{savage_cse_2019} \citep{kessler_overview_2016}. 
Finally, another mode that uses block ciphers is called the Message Authentication Code. Rather than encrypting data, the primary goal of this mode is to protect the integrity
and authenticity of the encrypted message \citep{anderson_security_2008}. MAC relies on CBC mode by only using the last block of the resulting cipher, 
the other blocks must remain secret \citep{anderson_security_2008}. 

    \begin{table}[]
    \centering
    \begin{tabular}{|c|c|c|c|c|}
        \hline
        c & r & y & p & t \\
        \hline
        o & a & b & d & e \\
        \hline
        f & g & h & i & k \\
        \hline
        l & m & n & q & s \\
        \hline
        u & v & w & x & z \\
        \hline
    \end{tabular}
    \caption{Grid for the key}\label{tab:key}
    \end{table}

\subsection{Three Types of Cryptographic Algorithms}

There are three main families of cryptographic algorithms: Secret Key Cryptography (or symmetric cryptography), Public Key Cryptography (or asymmetric cryptography), and
Hash Functions \citep{kessler_overview_2016} \citep{savage_cse_2019}. Symmetric cryptography uses one key to encrypt and decrypt data, therefore both the sender and the 
receiver must know the key which is secret, and distributing it becomes the main difficulty \citep{kessler_overview_2016}. Keys are defined as binary strings, 
and their space is expressed in bits \citep{savage_cse_2019}. 
For example, if someone would want to find a 128-bit key using brute force, it will require 2\textsuperscript{128} attempts. 
Symmetric cryptography algorithms can be split into two groups: stream ciphers and block ciphers \citep{kessler_overview_2016}, 
these two schemes have been discussed in the previous paragraphs. 
Due to time, only two commonly used Secret Key Algorithms will be presented: Data Encryption Standard (DES), and Advanced Encryption Standard (AES) \citep{kessler_overview_2016}. 
DES was invented by IBM in the 1970s, and is a block cipher algorithm using 64-bit blocks and 56-bit keys \citep{kessler_overview_2016}. 
DES is no longer considered secure today, however there are two variants to make DES make more secure: 3DES and DESX \citep{kessler_overview_2016}.
AES is a successor to DES, and it uses a scheme called Rijndael invented by Belgian cryptographers Joan Daeman and Vincent Rijmen \citep{kessler_overview_2016}.
It is a block cipher algorithm that does not have a fixed size block length and key length; the key length could be either 128, 192, or 256 bits;
and the block length could be also 128, 192 bits, or 256 bits \citep{kessler_overview_2016}. AES is used in BitLocker which is a hard drive encryption software \citep{noauthor_bitlocker_2019}.

\begin{figure}[]
    \centering
    \includegraphics[width=\textwidth]{images/SymEnc.PNG}
    \caption{This figure presents how symmetric cryptography works.
     The identical colour of the key indicates that the same key is used for encryption and decryption 
     \citep{kessler_overview_2016}.}
    \label{fig:sym}
\end{figure}

Public Key Cryptography relies on two keys: the public key and the private key. 
A public key is used to send some encrypted plaintext of fixed size that can only be decrypted with the private key \citep{savage_cse_2019}. 
Also, a public key can be used to verify that a message was sent by the owner of the private key, 
and hence the owner can use his private key to digitally sign a message to provide authenticity \citep{savage_cse_2019}.
Asymmetric cryptography relies on one-way functions: these are functions for which an inverse is hard to compute \citep{kessler_overview_2016}.
One type of one-way function is multiplication and its inverse is factorization: suppose there are two prime numbers 7 and 13, and their product is 91; 
to take the inverse of this operation, all that is known is that the number is 91 and it factors out into two prime numbers \citep{kessler_overview_2016}.
Therefore, computing a product is much faster than factoring a number into prime numbers \citep{kessler_overview_2016}. 
This whole idea relies on the fundamental theorem of arithmetic: 
"Each natural number greater than 1 factors into prime numbers in a way that is unique up to the order of the factors" \citep{anderson_security_2008}.
The second type of one-way function is exponentiation and its inverse, logarithm \citep{kessler_overview_2016}. Suppose computing 2\textsuperscript{5} which is equal
to 32, then computing the inverse requires to find two integers, x and y such that log\textsubscript{x} 32 = y. This computation requires more time than the exponentiation \citep{kessler_overview_2016}.
RSA is the most common public key cryptographic algorithm invented by Ronald Rivest, Adi Shamir, and Leonard Adleman in 1978, 
and it depends on the hardness of factorization \citep{kessler_overview_2016} \citep{savage_cse_2019}.
RSA is used in web browsers for secure communication with websites (SSL and its successor TLS) \citep{anderson_security_2008}.

\begin{figure}[]
    \centering
    \includegraphics[width=\textwidth]{images/AsymEnc.PNG}
    \caption{This figure presents how asymmetric cryptography works.
    The different key colours indicate that two different keys are used for encryption and decryption 
    \citep{kessler_overview_2016}.}
    \label{fig:asym}
\end{figure}

Hash Functions are used to create a string of fixed size from some input of arbitrary length \citep{savage_cse_2019}. 
They have two characteristics: knowing the hash, it must be difficult to find an input to the hash function so that it matches the hash;
it must also be difficult to find two inputs that generate the same hash \citep{savage_cse_2019}. 
Hash functions rely on block ciphers to work and are used in a similar fashion as in symmetric cryptography \citep{anderson_security_2008}. 
Hash functions are used to ensure the integrity of a file, and are commonly used to encrypt passwords \citep{kessler_overview_2016}. What follows are two currently used hash functions. 
Due to the nature of this paper, only two families of hash function will be presented: Message Digest (MD) and Secure Hash Algorithm (SHA).
Message Digest algorithms (MD) are hash functions that take an arbitrary-length string and generate a 128-bit hash \citep{kessler_overview_2016}.
There are several versions of this hash function (MD2, MD4, MD5) and collision vulnerabilities were found in the MD5 algorithm \citep{kessler_overview_2016}.
Collisions occur when two different plaintext generate an identical hash and this violates one of the characteristics of hash functions.
The Secure Hash Algorithm (SHA) is a hash function defined by the National Institute of Standards and Technology (NIST) \citep{kessler_overview_2016}. 
SHA-1 is the first version of SHA and it produced a 160-bit hash, it was broken in 2015 \citep{kessler_overview_2016}. SHA-2 relies on five different algorithms
SHA-1, SHA-224, SHA-256, SHA-384, and SHA-512 to generate hash values of 224, 256, 384, or 512 bits. 
SHA-3 (or Keccak) is a hash function that does not rely on its predecessors as the NIST found it necessary to have a different algorithm in case SHA-2 is broken \citep{kessler_overview_2016}.
Finally, bcrypt is a hash function that has been specifically designed for password hashing \citep{noauthor_1999_nodate}. 
\citet{savage_cse_2019-2} claims that this hashing algorithm should be used among others for storing passwords as it is slower than SHA-2 for example.

Finally, each of these algorithms has a specific purpose in cryptography.
Hashing is usually used fingerprinting data to ensure that it has not been modified \citep{kessler_overview_2016}.
In instance, if the receiver does not compute the same hash as the sender of a message, 
this means that the message has been modified. 
Symmetric cryptography is used to encrypt messages \citep{kessler_overview_2016}. 
The sender and the receiver of the message will have to use the same key.
Asymmetric cryptography is used for key exchange \citep{kessler_overview_2016}.
The sender uses the receiver's public key to transmit the key used to encrypt the message.
The receiver then uses his private key to decrypt the key, and thus the message.

\subsection{Mistakes in Cryptography}

Cryptography is an important part of a secure computer system. However, mistakes are made when these algorithms are integrated in a system,
and this is usually the spot where cryptography fails \citep{lazar_why_nodate}. This section will first investigate why cryptography itself is not the issue in most cases.
Then, it will discuss the mistakes that are made, and how they could be recovered from.

Cryptography is often viewed as a panacea. Focusing only on cryptography is not the way to go as that enables attackers to go around it \citep{schneier_security_nodate}.
However, it is powerful when the whole system is designed correctly \citep{schneier_security_nodate}. 
\citet{schneier_security_nodate} claims that most of the time attackers will not attack the cryptography itself, but they will more go after the vulnerabilities around the
implemented cryptography. This is mostly due to attackers not having the skills, the infrastructure, or the time to break the cryptography. 
That said, a cryptographic algorithm relying on its secrecy is also weak \citep{schneier_security_nodate}. It could take some days at most for specialists to reverse engineer 
the algorithm and discover the plaintext. One of the mistakes made in cryptographic system design is the failure to destroy the plaintext after encrypting it. 
\citet{schneier_security_nodate} and \citet{lazar_why_nodate} mention this issue, and it is a mistake where cryptography is not to blame, but the implementation is.
Another mistake people make is the poor usage of random numbers or not using them at all. \citet{lazar_why_nodate} state that random numbers are essential for cryptography, 
which could be used to generate a secret key. Sometimes, the random number generator is not certified for cryptography, and this allows attackers to predict the next bits.
It can also happen that the seeds for the random number generator are not random which also allows to predict the resulting key \citep{lazar_why_nodate}.
Attackers are often helped by the choice of weak keys. Joining the idea of using good random numbers to generate a strong key, some people prefer to bypass randomness,
and set weak passkeys generated by users that do not have a background in security as guessing a password if it is a key is much easier than trying to brute-force a key \citep{schneier_security_nodate}.
However, if people are asked to use strong passwords, they will forget them \citep{schneier_security_nodate}.
\citet{lazar_why_nodate} also claim that some developers hard-code the secret keys into the source code which cancels out the purpose of cryptography.
They think it is due to developers believing compilation will hide the key which is not true.
The use of outdated or weak algorithms (like MD4, MD5, SHA-1) is a problem that has been described by \citet{lazar_why_nodate}.
The vulnerability of MD5 has been described in a previous paragraph of this section.
They claim that this is mostly due to developers using past practices that are no longer considered safe, and using default settings in cryptographic libraries.
\citet{acar_comparing_2017} asked some coders to write some code using one of the five Python cryptographic libraries.
They came to the conclusion that some cryptographic libraries are poorly documented, one of them provided insecure symmetric encryption for example.
Therefore, good documentation for cryptographic APIs is needed for it to be used.
They also think that the easier it is to use a cryptographic library, the better the security results are.
Finally, they noticed that some developers said that their code is secure when it was not. This shows a lack of recognition for some security issues \citep{lazar_why_nodate}.

\subsection{Conclusion}

In conclusion, this section provided a small history of cryptography, some cryptographic fundamentals and algorithms. 
As most students taking the Cyber Security Fundamentals (presented in the \ref{sec:Introduction}) do not know much about cryptography, 
it is essential to have some understanding about cryptography so that the resulting game replicated the real world.
Finally, it explored issues that occur when cryptography is implemented, and why there is a need to educate people in cryptography.
In the previous paragraph, the following mistakes were identified in different research papers:
\begin{itemize}
    \item Cryptographic algorithm relying on its secrecy;
    \item Failure to delete the plaintext;
    \item Poor use of random numbers;
    \item Use of weak keys;
    \item Use of outdated algorithms;
    \item Poor choice of cryptographic APIs;
    \item Poor general security awareness.
\end{itemize}
All the mistakes presented are due to wrong choices made by developers when they were implementing it in their systems.
\citet{eggleton_value_2001} claim that problem solving is a process that requires iterations, and each iteration allows
people to reach a better solution by learning from their previous iterations. 
Therefore, the above mistakes present to students an early solution to a problem, show that they are not optimal, and
finally present a better solution to that problem. 
The next section will analyse some cyber security games, and demonstrate why games should be developed for cryptography.

\section{Cybersecurity Games Analysis}

This section will present ten cybersecurity games. These games will be reviewed according to their gamification or serious game characteristics,
and their cryptography content. Links to these games will be found in the appendix of this paper. 
At the end of this section, a small summary will be provided of all of the games reviewed.

\subsection{CyberCIEGE}

CyberCIEGE is "an innovative video game and tool to teach computer and network security concepts" developed by the Naval Postgraduate School and 
sponsored by the United States Department of the Navy \citep{noauthor_cyberciege_nodate}. The player takes the role of a cybersecurity specialist, 
and he is tasked to secure the network of a company. The player has a bird-eye view of the office, and his goal is to secure the network.
The game presents seven campaigns. Due to the nature of this paper, the main focus was accorded to the "encryption" campaign.
In one of the "encryption" campaign scenarios, the player is tasked to set up an encrypted messaging solution so that employees could communicate sensitive documents securely.
However, cryptography is mentioned at a very high-level, this challenge does not go further in depth than symmetric and asymmetric cryptography as an example.
If the player fails or succeeds to provide a secure way of communication, he gets feedback about what he did right and wrong.
While completing this challenge, the user is asked a couple of multiple choice questions about cryptography to reinforce the concepts learned in the challenge.

In summary, CyberCIEGE is a game that presents some gamification characteristics, and covers cryptography implementation at a high-level.

\subsection{Targeted Attack}

Targeted Attack is a game developed by TrendMicro, and it covers general cybersecurity concepts.
The player takes the role of Dave, the CIO of a company called "Fugle Incorporated" which is about to release a "revolutionary payment application".
The player watches some videos, and has to answer multiple choice questions to make the application as secure as possible. 
Each answer to a question is a decision, and each decision will affect the overall outcome. 
For some decisions, the player is also influenced by the financial constraints of his decisions just like in the real world.
One of the decisions mentions encryption, and thus cryptography: the player is asked whether he should encrypt the intellectual property of the company, 
or just remain compliant with the regulations. Choosing encryption costs the player two coins. 
However, encrypting the intellectual property is not the solution to the attack as the attackers are not after that.
This demonstrates that cryptography does not solve a problem if the threat model is incorrect.
At the end of the game, the player is told whether he succeeded. If he failed, the game would go over the mistakes the player made, 
and is given the opportunity to play again to set things right.

In broad terms, Targeted Attack is a gamification that presents some cybersecurity concepts and limited cryptographic concepts.

\subsection{The Weakest Link}

The Weakest Link is a game developed by IS Decisions. The player is asked thirty multiple choice questions about general workplace cybersecurity.
Topics covered include password management, movement of corporate files. The goal for the player is to obtain the highest possible security score.
There are no cryptography questions.

In conclusion, The Weakest Link is a gamification that could be given to new hires in a company to promote healthy security practices.

\subsection{Cyber City Demo}

Cyber City Demo is a game developed by Jason Stanton for Cyber Security Challenge UK.
The game is situated in a fictitious city called Cyber City, and presents open world game play resembling to Pokemon for Nintendo.
This game familiarizes the player with the Computer Misuse Act by making some people stand in front of a court for some computer misdemeanors they did.
The player has to determine if the accused are guilty of a crime, and the player gets feedback on their judgment.
Another challenge in this game demonstrates why public WiFi networks might not be secure. The player is tasked with determining the person who is spoofing a cafe WiFi.
At the end of this challenge, the player receives some feedback about how they accomplished the challenge.
A third challenge for the player is to determine whether some emails are phishing email. At the end, user gets feedback.
The last challenge is to setup a firewall in a factory. The player is introduced to how a firewall works, and sets up some firewall rules.

In summary, this game has some gamification concepts, introduces players to some cybersecurity concepts, but does not cover cryptography.

\subsection{Nite Team 4 (Demo)}

Nite Team 4 is a game developed by Alice \& Smith, and provides a UNIX style operating system interface. 
The player is introduced to some cybersecurity tools used in the real world, 
and allows the player to develop their skills in hacking while exploring at the same time the world of cyber warfare.
However, as the demo version has very limited features, nothing more can be said about this game in the sense of content.

In summary, this game resembles a lot to a serious game as it contains a real UNIX terminal and software used by cyber security professionals, 
however due to limitations, more cannot be said.

\subsection{CIA Cryptography Game (Break the Code)}

This game was found on the United States Central Intelligence Agency website. 
It is a small game for children where the player has to manually decrypt a substitution cipher. 
As a reward, the player gets to know the plaintext and its meaning after decryption.
Unfortunately, the website was no longer accessible when this section was written. 

In essence, this game is a gamification, and it presents one cryptography component, substitution ciphers. 

\subsection{University of Edinburgh Cyber security Game}

This game is a firewall setup game. It is based upon the "iptables" software in Linux. 
The player has to set up some rules in the firewall so that two users can communicate securely over the network.
Some information, like ip addresses, applications that are being used, is given to the player so that he can setup the rules correctly.
The player is given some choices which are arguments to the "iptables" command so that the rules are set up correctly.
If the player fails to set them up correctly, the player is told which rules are correct, and which are not. 
He is then given the opportunity to correct his set up after the feedback.

Ultimately, this game presents some characteristics of a Serious Game, however it is not relevant to cryptography.

\subsection{Data Centre Attack}

Data Centre Attack is another game by TrendMicro, and it follows a similar interaction scheme as the Targeted Attack game mentioned in section 2.3.1.
The player takes the role of Mark, a computer information security officer in a hospital. 
The player has to make the right decisions in order to protect the hospital from data breaches and ransomware.
At the end of each video which provides context and guidance to the player, he must take a decision to prevent a data breach in the hospital.
Each choice has an influence on the outcome of the game, and there are some financial constraints just like in the real world.
At the end of the game, the player is walked through the decisions he made, and is given feedback on what each decision meant.
Unfortunately, the game does not introduce any cryptographic concepts.

In a nutshell, this gamification does not present any cryptographic content.

\subsection{Cyber Challenge}

Cyber Challenge is a game designed for the United States Department of Defense. This game presents three challenges.
The first challenge is a high-level network design where the player has to correctly place the firewall, server, database, and storage components.
The second challenge is about determining whether some user actions on the network are malicious.
The third challenge involves calculating IP addresses in binary. 
At the end of each challenge, the player is given feedback about their answers.

Finally, this game is a gamification which unfortunately does not present any cryptographic content.


\subsection{Cyber Lab}

Cyber Lab is a game found on the website of the Public Broadcasting Service (PBS). 
In the beginning of the game, the player chooses his avatar and the startup he works for.
The user has to complete three challenges. 
The first challenge is a "coding challenge": the player has to write some code using a puzzle to move a character across a path.
The second challenge is a "password cracking challenge": the player has to set up a password so that it cannot be found out, 
and must at the same time crack his opponents password using common sense, brute force, and dictionary attacks.
The third challenge is a "social engineering challenge": the player is tasked with determining whether an email is a phishing email, 
and, if it is, showing what characteristics makes that email a phishing email.
Each challenge earns the player some points which are used to invest in cybersecurity defenses to protect the startup.

In summary, this game has some gamification characteristics, but does not have any cryptography content.

\subsection{Summary}

This section presented ten cybersecurity games, they have been summarized in Table \ref{tab:games}.
The main fact to notice is that there are not many games that cover cryptography, and none that cover cryptography in depth.
This shows that there is a lack of games about cryptography, 
and shows that there is a gap in the market where an educational game about cryptography could fit in.

\begin{table}[]
    \centering
    \resizebox{\textwidth}{!}{%
    \begin{tabular}{|l|l|l|}
    \hline
    \textbf{Game Name}           & \multicolumn{1}{c|}{\textbf{Gamification or Serious Game}}                                                                                      & \multicolumn{1}{c|}{\textbf{Cryptography Content}} \\ \hline
    CyberCiege                   & \begin{tabular}[c]{@{}l@{}}Gamification:\\ Make choices and buy infrastructure \\ to make a system secure \\ in a simulated office\end{tabular} & High-level (symmetric and asymmetric)              \\ \hline
    Targeted Attack              & \begin{tabular}[c]{@{}l@{}}Gamification:\\ Make choices to create a secure version of an app\end{tabular}                                       & Mentioned once (high-level)                        \\ \hline
    TheWeakest Link              & \begin{tabular}[c]{@{}l@{}}Gamification:\\ Multiple Choice Questions to make employees \\ more aware about security\end{tabular}                & None                                               \\ \hline
    Cyber City Demo              & \begin{tabular}[c]{@{}l@{}}Gamification:\\ Inform people about the daily \\ cyber security risks\end{tabular}                                   & None                                               \\ \hline
    Nite Team 4                  & \begin{tabular}[c]{@{}l@{}}Serious Game:\\ Train people into using real life \\ cyber security tools\end{tabular}                               & Only demo version played, not enough to conclude   \\ \hline
    CIA Cryptography Game        & \begin{tabular}[c]{@{}l@{}}Gamification:\\ Present cryptography fundamentals \\ to children\end{tabular}                                        & Introduction to substitution ciphers               \\ \hline
    University of Edinburgh Game & \begin{tabular}[c]{@{}l@{}}Serious Game:\\ Learn how to use the "iptables" program\end{tabular}                                                 & None                                               \\ \hline
    Data Centre Attack           & \begin{tabular}[c]{@{}l@{}}Gamification:\\ Make choices to secure the data of a hospital\end{tabular}                                           & None                                               \\ \hline
    Cyber Challenge              & \begin{tabular}[c]{@{}l@{}}Gamification:\\ Learn the basics of network security\end{tabular}                                                    & None                                               \\ \hline
    Cyber Lab                    & \begin{tabular}[c]{@{}l@{}}Gamification:\\ Learn the risks of weak passwords and phishing \\ through simple games\end{tabular}                  & None                                               \\ \hline
    \end{tabular}%
    }
    \caption{Summary of the Games}\label{tab:games}
\end{table}

\section{Conclusion}

This literature review and market research presented some important concepts which are going to be used to build the cryptographic part of the cybersecurity game.
The first section presented gamification and serious games, their characteristics and proof that they work. 
The second section presented some cryptographic fundamentals, algorithms, and mistakes which are going to be used in the cryptography quest line.
Finally, the last section presented ten cybersecurity games, and they were analysed on their cryptographic content.
These games were found after an extensive week of google searches, and they were chosen because of their content in either gamification or serious games, and cyber security. 
This section demonstrated that more can be done in this field, and justifies the project approach.

%==================================================================================================================================
\chapter{Analysis/Requirements}

Lectures tend to lack interaction and some questions that the students might have, are left unanswered \citep{lehmann_theory-driven_2014}.
Moreover, as a high number of students tend to attend lectures, it is impractical to give feedback concerning their understanding of a subject 
which is an important factor in learning \citep{lehmann_theory-driven_2014}.
Hence, the idea is to supplement the lecture material in the Cyber Security Fundamentals with an open world game that would test students' understanding
of topics presented in this course.
In early 2019, a group of students completing their master's degree in cyber security decided to build a game that would go
beyond the material presented in the Cyber Security Fundamentals Course.
This game is an open-world game, and features the main character called "Mike the Panda" who learns cyber security concepts by
interacting with various characters in the city. 
The students built challenges in the following subjects which were primed during the lectures:
\begin{itemize}
    \item Web Applications Attacks (XSS, CSRF);
    \item Penetration testing;
    \item Cryptography;
    \item Digital Forensics.
\end{itemize}
Unfortunately, the challenges lacked cohesion, and some did not go beyond the material.
Lack of cohesion was due to the fact that each challenge had its own flow, and its own means of interaction with the user 
(for example, the key bindings varied between quests). 
Hence, it has been decided to maintain the already built map, to improve existing quests, and to construct new challenges where required.
Moreover, since this course presents cryptography in a limited way, and the principal goal is to give more material about this subject,
a cryptography quest is going to be built into the game. This would allow the students to learn more about it, 
and to receive feedback about what was covered in lectures.
This chapter will first present the Intended Learning Outcomes (ILOs) for the cryptography quest, and then will proceed towards a more specific 
formulation of those requirements.

\section{Intended Learning Outcomes (ILOs)}

The Intended Learning Outcomes are the following:
\begin{itemize}
    \item Go beyond the lecture material presented in the course;
    \item Present what cryptography is about;
    \item Present the three types of cryptographic algorithms;
    \item Demonstrate commonly made mistakes in cryptographic applications.
\end{itemize}

Each of these will be elaborated in the following sections in detail, 
and how these were established, will be explained.

\subsection{Beyond the lecture material}

As stated in the introduction of this chapter, lectures lack in interactions and feedback.
Hence, a core requirement of the game is to verify students' knowledge in cryptography, 
and to supplement it where possible. To understand what concepts should be supplemented, 
a quick overview of the cryptography material in the CSF is described.
The CSF course introduces students to cryptography basics.
It first introduces primitive algorithms like Caesar's cipher and the Vigenère cipher which are both
examples of stream ciphers. 
In the background section of this paper, these two algorithms have been presented.
It then presents and explains the three operations in cryptography: encryption, decryption, and hashing.
It then proceeds to demonstrate how symmetric and asymmetric cryptography work, what keys are used, and where they are applied.

This requirement is the parent of the other intended learning outcomes. 
Indeed, the question that has been asked for every idea that came to mind is the following:
"How does it improve the students' learning experience and what new concepts does it bring to the table?"
With this in mind, the other ILOs will be now presented.

\subsection{Purpose of cryptography}

The purpose of cryptography requirement is all about algorithms that have been used in the past to understand its purpose.
The core goal of this requirement is for the student to gain understanding what cryptography is about, and where it has come from.
A consequential goal of this requirement is to understand why these algorithms are not longer in use.
A good algorithm to fullfil these goals is the Caesar's cipher because it is easy to explain, 
understand, and to break as claimed by \citet{anderson_security_2008}.

Going in this direction, an exercise that will be proposed to the student is
the student receives a cipher and a number (the key) associated with it, and use these two pieces of information to decrypt the message.
To accomplish this, the students will have to complete the following tasks:
\begin{itemize}
    \item 1. Determine what algorithm was used to encrypt the message;
    \item 2. Determine what the key is to decrypt the message;
    \item 3. Decrypt the message itself.
\end{itemize}

\subsection{Three types of cryptographic algorithms}

The course presents the three cryptographic algorithm in use: symmetric cryptography,
asymmetric cryptography, and hashing. 
The goal of this requirement is to understand the purpose of each of these algorithms,
and why there are three of these. As mentioned by \citet{kessler_overview_2016},
symmetric and asymmetric cryptography complement each other. 
Symmetric cryptography is used for message transmission, and public key cryptography 
is used for key exchange. Hashing is used to store passwords and to verify file integrity.
Hence, this ILO will develop the idea of why there is a need for symmetric and asymmetric
cryptography, and how it is accomplished in practice.
Finally, this requirement will present to the student commonly used algorithms for each
type of cryptographic algorithm as this is not covered in detail in the course.

Since the course does not go into details about how all of these algorithms are 
linked together, the following requirements can be formulated:
\begin{itemize}
    \item Present symmetric cryptography, the keys it uses, and its commonly used algorithms;
    \item Present asymmetric cryptography, the keys it uses, and its commonly used algorithms;
    \item Present hashing, and its commonly used algorithms;
    \item Present how symmetric and asymmetric cryptography are combined together
    to establish secure communication.
\end{itemize}

\subsection{Mistakes in cryptography}

\citet{eggleton_value_2001} claim that learning from mistakes is important as it allows people to 
learn why their solutions were incorrect. Hence, presenting mistakes in 
cryptographic implementations would help people to see why these were incorrect.
This would in turn avoid people to make mistakes further on in their careers.
This fits in the main goal of supplementing the lecture material as these mistakes 
have not been explained in lectures.

In the Background chapter, some common mistakes were identified. 
Some of them are more likely to happen than others.
The following mistakes were chosen as requirements for this Intended Learning Outcome:
\begin{itemize}
    \item Cryptographic algorithm relying on its secrecy;
    \item Failure to delete the plaintext;
    \item Poor use of random numbers;
    \item Use of weak keys;
    \item Use of outdated algorithms.
\end{itemize}
Security awareness was not chosen because it is too broad of a subject to be implemented. 
Poor choice of cryptographic APIs was not chosen because there are many APIs available,
and because it requires a specific decision criteria to determine which APIs are better than others.
Since the latter point is beyond the scope of this paper, it has not been implemented. 

\section{Conclusion}

First, this chapter presented the existing framework in which the cryptography challenges will be
built. Then, it presented the main goal which to enhance and go beyond the material presented in lectures.
Finally, it presented the Intended Learning Outcomes and their more specific requirements.
They have all been summarized in table \ref{tab:requirements}.

\begin{table}[]
    \resizebox{\textwidth}{!}{%
    \begin{tabular}{|l|l|}
    \hline
    \textbf{Intended Learning Outcomes}                    & \textbf{Specific Requirements}                                                                                                                                                                                                                                                                                                                                                                            \\ \hline
    Go beyond the lecture material presented in the course & This is the ILO that spawns off the the other ones                                                                                                                                                                                                                                                                                                                                                        \\ \hline
    Present cryptography origins                           & \begin{tabular}[c]{@{}l@{}}1. Determine what algorithm was used to encrypt the message;\\ 2. Determine what the key is to decrypt the message;\\ 3. Decrypt the message itself\\ .\end{tabular}                                                                                                                                                                                                           \\ \hline
    Present the three types of cryptographic algorithms    & \begin{tabular}[c]{@{}l@{}}1. Present symmetric cryptography, the keys it uses, \\ and its commonly used algorithms;\\ 2. Present asymmetric cryptography, the keys it uses, \\ and its commonly used algorithms;\\ 3. Present hashing, and its commonly used algorithms;\\ 4. Present how symmetric and asymmetric cryptography \\ are combined together to establish secure communication.\end{tabular} \\ \hline
    Mistakes in cryptography                               & \begin{tabular}[c]{@{}l@{}}1. Cryptographic algorithm relying on its secrecy;\\ 2. Failure to delete the plaintext;\\ 3. Poor use of random numbers;\\ 4. Use of weak keys;\\ 5. Use of outdated algorithms.\end{tabular}                                                                                                                                                                                 \\ \hline
    \end{tabular}%
    }
    \caption{Summary of the Intended Learning Outcomes and their specific requirements}
    \label{tab:requirements}
\end{table}

%==================================================================================================================================
\chapter{Design}

The previous chapter analysed the core problem, presented the important requirements for the challenges,
and described the existing context in which the game is built. 
Since there are three major concepts in cryptography which have to be developed, 
the cryptography quest line will be defined in three different challenges. 
The first challenge will be about primitive stream ciphers; 
the player will receive an encrypted message, and will have to decrypt it using 
Caesar's cipher or the One-Time Pad cipher. 
The second challenge will test the student's knowledge of cryptographic algorithms;
the student will have to design on a high level secure cryptographic systems.
The final challenge will introduce the student to cryptographic mistakes; 
he will have to browse a file system to find files that have been compromised due to poor use off
cryptography. All these challenges are to be completed in the order they are presented
and will be linked together with an in-game character called "Crypto Jedi". 
Finally, all challenges have been designed with \citet{cugelman_gamification:_2013}'s characteristics
presented in the background chapter.

\section{Challenge 1}

\subsection{Storyline}

This challenge kicks off the cryptography quest line. It starts with Mike (the player character) 
meeting Crypto Jedi at home. Crypto Jedi asks Mike to decrypt a message he received. Mike accepts, goes
to a laptop to read the email, and decrypts the message. Once decrypted, 
the message instructs the player to go to the coffee shop where the second challenge awaits.

\subsection{Challenge Design}

This challenge has two important design parts. The cipher message design and the "decryptor" design.
This challenge is about primitive stream ciphers. Hence, the challenge was designed in such way that
the student would have the choice between Caesar's cipher and the One-Time Pad cipher. Eventually,
the message was encrypted with Caesar's cipher because the One-Time Pad uses the xOR operation which requires
characters being rewritten in their bit representation using the ASCII scheme. 
Hence, there is a possibility that some characters obtained after encryption cannot be mapped in their ASCII form.
However, the xOR encryption was kept in the final design of the challenge to give the students some exposure
to this algorithm and to add some "noise" to the challenge so that it is not too easy.
Finally, it was decided that the challenge would be worth 100 points, and each time the user would input
a wrong answer, he would lose 25 points.

Once these points were clarified, the message to be decrypted had to be written. 
The message has to fit in with the storyline described above. 
Thus, it was decided that the message will be the following: "Meet me at the coffee shop. Crypto Jedi".
Afterwards, the message was encrypted with Caesar's cipher to obtain this cipher: "TLLA TL HA AOL JVMMLL ZOVW. JYFWAV QLKP."
(alphabet is shifted by 7 characters to the right). 
Finally, the cipher and the key was added to the email which the player will see.
There was discussion whether the key should be added to the email 
or the player has to figure it out himself. 
Eventually, it was kept because it was deemed that the challenge might become too difficult for the student.

Finally, the "decryptor" had to be designed. Since both Caesar's cipher and the One-Time Pad are stream ciphers,
they can hence be encrypted and decrypted character by character. 
Therefore, an interface was designed in such way that the player can see how the algorithms work.
For the Caesar cipher, the user would input a character, and a shift key. 
In turn, he would see how the alphabet was shifted, and what the resulting character is. Eventually,
the user would decrypt the message character by character. 
Once the message is encrypted, he would have to input it in a dedicated text box to get the message validated
and obtain the points. 
For the xOR cipher, the same philosophy was used. The user would input a character and a key (a number).
Then, he would see the character converted in bits, the key converted in bits, and the resulting character.
An answer text box was added to the interface so that the user can input their answer even though
the One-Time Pad algorithm is not the correct solution to this problem.
After the challenge is completed, 
the user gets some explanation about stream and block ciphers and how they function.
Block ciphers have been added so that the student knows that stream cipers are not 
the only type of primitive cryptography.
Figure \ref{fig:Caesar} presents the wire frame for the Caesar cipher 
and Figure \ref{fig:xOR} presents the wire frame for the One-Time Pad.
\begin{figure}
    \centering
    \begin{subfigure}[b]{0.5\textwidth}
        \includegraphics[width=\textwidth, frame]{images/CaesarCipher.png}
        \caption{Wire frame for the Caesar cipher decryptor interface.}
        \label{fig:Caesar}
    \end{subfigure}

    \begin{subfigure}[b]{0.5\textwidth}
        \includegraphics[width=\textwidth, frame]{images/xOR.png}
        \caption{Wire frame for the One-Time Pad decryptor interface.}
        \label{fig:xOR}
    \end{subfigure}
    
    \caption{Wire frames for the first challenge}
    \label{fig:Challenge1}

\end{figure}

\subsection{Gamification Characteristics}

This section will examine how this challenge satisfies the gamification characteristics identified
by \citet{cugelman_gamification:_2013}.
\begin{itemize}
    \item "Goal-setting": The goal of this challenge is to decrypt the cipher. 
    A higher-level goal of this challenge is to find the meaning of the message, gain points by decrypting it correctly, 
    and to advance the cryptography quest.
    \item "Capacity to overcome challenges": The challenge has been designed in such way 
    that it is not too difficult for the student by including the key to the cipher for example.
    \item "Providing feedback on performance": When the student thinks he completed the challenge and submits his answer,
    he receives feedback whether he decrypted the message correctly, 
    and the number of points he obtained if the message is decrypted successfully. 
    The explanation at the end of the challenge provides some supplementing feedback, 
    and allows the user to retry if he wishes to do so.
    \item "Reinforcement": The number of points and feedback reinforce whether the student understood how streams ciphers work.
    \item "Progress comparison": The points system opens the door for a potential discussion and comparison between students.
    \item "Social connectivity": Students could potentially interact with each other to solve the challenge.
    \item "Fun and playfulness": The open-world game enables players to discover a new world, 
    and decrypting the message might be considered as a fun activity by some.
\end{itemize}
As this challenge satisfies the above points, 
it can be implemented and evaluated according to these points.

\section{Challenge 2}

\subsection{Storyline}

This challenge starts off by Mike meeting the Crypto Jedi at the coffee shop after completing the first challenge.
Crypto Jedi tells Mike that he was asked by a company to design a secure messaging service. 
Crypto Jedi asks Mike to design the sign in, send message, and receive message functionality
using correct algorithms. Once he completes the challenge, Mike should meet Crypto Jedi at the office.

\subsection{Challenge Design}

This challenge is split into three distinct parts: sign in, send message, and receive message.
All three challenges have been designed in a flowchart style. 
Each box contains either an input to an algorithm, the choice of an algorithm, 
or the output of the algorithm. 
The choice between algorithms remains the same throughout the three parts of the challenge.
The player has the choice between the following hashing algorithms: MD5, SHA-2, and bcrypt.
The player can also choose these secret-key algorithms: DES, and AES.
Finally, the player can only choose one public-key algorithm: RSA.
Every time the student chooses an algorithm requiring a key, 
he must choose an appropriate key for that algorithm. 
These details will be described in detail for each part of this challenge.
Finally, the whole challenge is worth 300 points, and each subsection is hence worth 100 points.

The sign in design task asks the user to find a secure way to validate or to store user passwords.
The user gets as input the user password 
and his goal is to ensure that the password is stored or validated securely.
To accomplish this, he has the choice between the algorithms described above, 
Also, he has the possibility to combine the password with either the username of a user,
or a salt that is unique to the password.
The correct solution to this problem is to choose bcrypt for the password hashing and
to combine with the password a unique salt.
If the player chooses an incorrect type of algorithm (symmetric or asymmetric), he loses 25 points.
If the player chooses MD5 as hashing algorithm, he would lose 15 points because MD5 is considered
as vulnerable. If the player chooses SHA-2 as hashing algorithm, he would lose 5 points because 
SHA-2 is not the optimal solution to this problem. 
At the end of the design exercise, the player is given an explanation about the solution to this problem.

The send message design task requires the student to transmit a message and its key securely using
the key exchange framework. 
The player receives as input a message, and can choose up to three algorithms. 
If they choose symmetric or asymmetric encryption, they will have a choice between the following keys:
recipient's private key, recipient's public key, sender's private key, sender's public key, 
sender's password, and randomly generated key. 
Figure \ref{fig:Challenge2} presents a solution to this exercise. To obtain this solution,
the student has to implement correctly the key exchange mechanism. 
They must generate a random key to encrypt the message using AES encryption, 
and then they must encrypt the key using RSA encryption with the recipient's public key.
If the player chooses an incorrect algorithm type, he loses 25 points. 
If the player chooses an incorrect key type, he loses 15 points.
If the player chooses the correct algorithm type, but it is an outdated one, he loses 5 points. 
Finally, this challenge also presents to the student another use of hashing, file integrity verification.
This can be accomplished by mapping the input message into a hash function.
However, the player will not be deducted any points if he does not do this 
as this is beyond the scope of the exercise.
At the end of this part, the student is debriefed concerning the task and what should have been done.
A disclaimer is also provided so that the student does not implement their own cryptographic algorithm.
This is to address the mistake of creating one's own cryptographic algorithm mentioned in the requirements chapter.
He is also given the opportunity to retry the challenge.

The receive message design task asks the player to do the opposite of the second part which
is to decrypt the received message. The key exchange framework is still used.
The player receives two inputs which are the encrypted message and the encrypted key.
His goal is to decrypt the message.
Solving this problem is almost similar to reversing the graph presented in \ref{fig:Challenge2}.
To achieve the correct solution, the player has to firstly decrypt the encryption correctly.
This involves using the RSA algorithm with the receiver's private key to decrypt the message encryption key.
Then, they must use that decrypted key to decrypt the message using AES encryption.
The same scoring scheme applies as in the second part of the challenge. 
At the end of this part, the student is given some explanations about the tasks of the challenge and
they are given the opportunity to retry the challenge.

\begin{figure}[b]
    \centering
    \includegraphics[width=\textwidth, frame]{images/Questline2Part2Cropped.pdf}
    \caption{Sample solution to the second part of the second challenge. 
    Specific algorithms are not specified.}
    \label{fig:Challenge2}
\end{figure}

\subsection{Gamification Characteristics}

This section will double check whether this challenge satisfies \citet{cugelman_gamification:_2013}'s
gamification characteristics.
\begin{itemize}
    \item "Goal-setting": The goal of this challenge is to design the cryptographic systems correctly. 
    A higher-level goal of this challenge is to gain points by completing the challenge, 
    and advance the cryptography quest.
    \item "Capacity to overcome challenges": The challenge has been designed in such way 
    that it is not too difficult for the student. There was an idea of making this exercise a coding exercise
    which would make this challenge more like a serious game than a gamification. 
    However, since most students taking CSF do not know how to code, it was decided to keep this format.
    \item "Providing feedback on performance": When the student thinks he completed the challenge and submits his answer,
    he receives feedback whether he designed the system correctly, some explanation is given about what should have been done,
    and allows the user to retry if he wishes to do so.
    \item "Reinforcement": The number of points and feedback reinforce whether the student understood how cryptographic algorithms work.
    \item "Progress comparison": The points system opens the door for a potential discussion and comparison between students.
    \item "Social connectivity": Students could potentially interact with each other to solve the challenge.
    \item "Fun and playfulness": The open-world game enables players to discover a new world, 
    and there could be some playfulness by advancing the quest.
\end{itemize}
Hence, this challenge satisfies the gamification characteristics.

\section{Challenge 3}

\subsection{Storyline}

After completing the second challenge, Mike joins Crypto Jedi at the office. 
There, Crypto Jedi asks Mike to investigate a hospital because there were claims of leaked confidentiality
patient data. This is the final challenge of the cryptography quest.

\subsection{Challenge Design}

In this challenge, the player has access to the hospital's data 
and his task is to find four files that prove that the hospital has been using bad cryptography practices.
This challenge will highlight the different mistakes presented in the requirements chapter.
The first file to be found is a file that has been encrypted, but the plaintext has been not been deleted.
Secondly, the user must find the random key generator in the file system, and test it out. 
He will eventually notice that it does not generate numbers randomly.
Then, he will also find another file that explains to the people using the system of the hospital
how to encrypt and decrypt files. There, the student will notice that the encryption algorithm used is 
DES which is considered as obsolete. 
Finally, he will notice a directory that contains an encrypted file and the key associated with this file.
The student will have to use the provided decrypting software to decrypt the file using the key.
Other files will be added to the simulated file system to provide some noise.

Once the player find an incriminating file, he must add it to the basket. 
Each time he adds a correct file to the basket, he must answer to a multiple choice question.
This multiple choice question will ask the student why he is adding the file to the basket.
They must hence choose the correct answer among the four possibilities. 
This is to avoid the player brute forcing the challenge and to ensure that they understood the material.
The whole challenge is worth 100 points. Each time the player selects an incorrect answer in the multiple
choice question he loses 10 points.

\begin{figure}[b]
    \centering
    \includegraphics[width=0.8\textwidth]{images/WireframeChallenge3.png}
    \caption{Wire frame for the third challenge}
    \label{fig:Challenge3}
\end{figure}

\subsection{Gamification Characteristics}

This section will double check whether this challenge satisfies \citet{cugelman_gamification:_2013}'s
gamification characteristics.
\begin{itemize}
    \item "Goal-setting": The goal of this challenge is to find mistakes in cryptographic implementations. 
    A higher-level goal of this challenge is to gain points by completing the challenge, 
    and finishing the cryptography quest.
    \item "Capacity to overcome challenges": The challenge has been designed in such way 
    that it is not too difficult for the student. 
    Students could have been penalized if they would try including every file to the basket, but that
    could have made the challenge to difficult for them.
    \item "Providing feedback on performance": When the student adds a file to the basket, he is asked
    a question about the reason why he added that file. Answering correctly to the question provides 
    feedback to the student whether he understood the reason correctly.
    \item "Reinforcement": The number of points and feedback reinforce whether the student understood the mistakes correctly.
    \item "Progress comparison": The points system opens the door for a potential discussion and comparison between students.
    \item "Social connectivity": Students could potentially interact with each other to solve the challenge.
    \item "Fun and playfulness": The open-world game enables players to discover a new world, 
    and there could be some playfulness by finishing the quest.
\end{itemize}
Hence, this challenge satisfies the gamification characteristics.

\section{Conclusion}

This chapter presented the high-level design of the three challenges.
The first challenge presents the cryptographic fundamentals to students by examining simple algorithms.
The next challenge develops concepts (symmetric, asymmetric cryptography and hashing) presented in lectures,
and allows students to test their understanding in these fields.
The final challenge introduces students to common cryptography mistakes and how they can be avoided.
Hence, the challenges fit the requirements established beforehand.
Finally, the challenges satisfy the gamification characteristics that were established in the background chapter of this paper.
The next chapter will present the implementation of these challenges 
and how they were connected to the the open world game.

%==================================================================================================================================
\chapter{Implementation}

To implement the design presented above, the implementation has been split into three steps.
The first step was to implement the challenges. The challenges were implemented in a web application
which used the React library \footnote{https://reactjs.org/} and some other small packages found on npm.
The second step was to implement the quest in the open-world game. 
The open-world game is built using Unreal Engine 4 \footnote{https://www.unrealengine.com/}. 
The final step was to connect the web application with the open-world game. 
To accomplish this, a backend was used to transfer information between the web application and the open-world game.
This backend was built using Express.js \footnote{https://expressjs.com/}.
The following sections will dive into more detail on the development of these three parts.

\section{Web Application}

The challenges were implemented in a web application. 
This allowed more flexibility and freedom than directly implementing these in Unreal Engine 4.
React applications are built with html, css, and javascript. 
This gives the developer more freedom by combining the structure, the look, and the logic together.
Certainly, Unreal Engine 4 provides a widget feature allowing the developer to implement simple user interfaces, 
but this feature is limited in functionality, 
and the design presented above is easier to implement in a web application.
Finally, Unreal Engine 4 provides the developer with a limited, but functional web browser.
This key feature allows the development of the challenges in a web application.
The next paragraphs will present the development of the challenges, and some interesting details concerning them.

The first challenge has the following flow. 
The player would first see the email that Crypto Jedi sent him. 
This email was implemented as a screenshot from an email provider and pasted into the web application.
Once they read the email, they proceed to the "decryptor" interface. 
The "decryptor" interface was designed as a form. 
Figures \ref{fig:CaesarImp} and \ref{fig:xORImp} are screenshots of the implementation of this "decryptor".
The user types in a character and key in their respective fields. 
Then, he must press the "decrypt" button to get the decrypted character.
Initially, the "decrypt" button was not part of the design. 
It was planned that the whole operation would happen without the need for the user to press a button.
However, since each text box has an event listener associated with, 
each time a character would be modified in these text boxes, an event would be triggered. 
This event would call React's \textit{setState} \footnote{https://reactjs.org/docs/react-component.html\#setstate} function 
which would store the contents of these text boxes in variables.
\textit{setState} is an asynchronous function which means that code after the call to that function would be executed 
before it returns a value. Hence, there was a bug where the alphabet would be shifted by one number less than the user typed in.
This is the reason why a "decrypt" button has been added which would call setState once everything has settled. 
Once the user has submitted the message into the answer text box, the text is validated, 
and the user gets a pop up text box telling them whether they completed the challenge successfully and the points they obtained.
Finally, they are redirected to a page where they can read through the explanation of the challenge, or return to the "decryptor".

\begin{figure}
    \centering
    \begin{subfigure}[b]{0.65\textwidth}
        \includegraphics[width=\textwidth, frame]{images/CaesarImplementation.PNG}
        \caption{Implementation of the Caesar cipher exercise.}
        \label{fig:CaesarImp}
    \end{subfigure}

    \begin{subfigure}[b]{0.65\textwidth}
        \includegraphics[width=\textwidth, frame]{images/xORImplementation.PNG}
        \caption{Implementation of the One-Time Pad exercise.}
        \label{fig:xORImp}
    \end{subfigure}
    \caption{Screenshots of the implementation of the first challenge.}
    \label{fig:Challenge1Imp}
\end{figure}

The second challenge asks the player to design on a high-level cryptographic systems.
It was divided in three parts as described in the design chapter.
For each part, the player would see three pages. 
The first page they would see is the directives to what they should do.
The next page is the challenge itself. 
The final page provides some explanation concerning the challenge and allows the user to try again if he wishes to do so.
Figure \ref{fig:Challenge2Imp} presents the implementation of the second part of this challenge with its solution.
The challenge is implemented as a flowchart as mentioned in the design chapter.
The user can choose key types and algorithm types using the dropdown elements as seen on figure \ref{fig:Challenge2Imp}.
An algorithm choice might reveal another element in the flowchart which might be the key to solve the challenge.
The arrows were implemented using Scalable Vector Graphics (SVG), and some of them appear or disappear according to the player's choices.
Finally, when the student has committed to a solution, a pop up box would appear telling the user whether they were correct.
If they were not, an explanation is provided, and they are invited to try again.
The final page provides a walk-through through the solution and explanation of the different algorithms used.

\begin{figure}[b]
    \centering
    \includegraphics[width=\textwidth, frame]{images/Challenge2Imp.PNG}
    \caption{Screenshot of the implementation of the second part of the second challenge.}
    \label{fig:Challenge2Imp}
\end{figure}

The final challenge is about finding mistakes in cryptographic implementations.
It was designed using the npm "react-keyed-file-browser" package which was used to implement the simulated file browser.
This package provides a convenient structure to display directories and files.
Each time the player would click on a file, it would "open" the file and show it as a popup.
Each file is implemented as a separate React component.
The player would then have the possibility to either add the file to the basket, or to close it.
Figure \ref{fig:Challenge3Imp} shows this feature. 
In the background of this figure, the file explorer can be seen. 
If the player decides to add the file to the basket, 
he will see a multiple choice question where he will have to choose the correct answer.
Once the correct answer is chosen, the file is shown in the basket, and all the popups are closed.
The user can also open "programs" using this file browser.
There are two such "programs": key generator, and the file decryptor.
The purpose of the key generator is to generate random keys.
However, it is implemented in such way that it would circle over five keys, 
hence not being random. This one of the files that can be added to the basket.
The purpose of the file decryptor is to decrypt files. 
The program contains two dropdown menus: the first one is used to select the file to decrypt,
and the second one is used to select a key among the provided choices. 
This "program" can be used to decrypt a file once the user finds the file containing the key.
This decrypted file will then appear in the simulated file system and can be subsequently added to the basket.
Once the user finds the four files, a popup would appear indicating the user he completed the challenge, and the points he scored.

\begin{figure}[b]
    \centering
    \includegraphics[width=\textwidth, frame]{images/Challenge3Imp.PNG}
    \caption{Screenshot of the implementation of the third challenge.}
    \label{fig:Challenge3Imp}
\end{figure}

\section{Connecting everything together}

Once the web app was implemented, it had to be connected to the the open-world game.
This involved three steps. The first step was to put into the map the Crypto Jedi character and the laptops 
where the user was going to do the challenges. Then, the web browser widget had to be implemented. 
The final step was to transport the challenge data from the application to the open-world game.
The next three paragraphs will describe these three steps in detail.

There were three places where a laptop and the Crypto Jedi had to be placed: Mike's home, the coffee shop, and the office.
For each of these places, the character and the laptop were placed next to each other.
Once these elements were placed, collision boxes had to be designed.
Collision boxes are events listeners which wait for the user character to walk into them so that an event can be triggered.
One collision box was added to each element. 
The collision box associated with the Crypto Jedi is used to trigger the start of the dialog.
The dialog was implemented in a widget and consists of a couple of sentences.
To trigger the dialog, the user has to be inside the collision box and then press the "F" key.
The other collision box was added to the laptop. 
This laptop is used to trigger the challenges, and guide the user towards the web application.
This point will be more developed in the next paragraph.

The laptop is associated with a web browser widget. 
The web browser widget \footnote{https://docs.unrealengine.com/en-US/Engine/UMG/UserGuide/WidgetTypeReference/WebBrowser/index.html} 
is an experimental feature present in Unreal Engine 4.
This feature allows the developer to point this web browser to any URL on the internet.
The only features present in this web browser are the following: set initial url, and load url.
Set initial url allows to set the url where the web browser will navigate first when it is opened.
Load url allows the developer to specify a new url to which the web browser will navigate once it is opened.
For the purpose of this project, only set initial url was used as the player only needs to navigate to the first page of each challenge.
Navigation between pages in a challenge are provided through clickable links in the webpage.
Once the player decides to start the challenge (by entering into the collision box and pressing the "F" key), 
the widget fills all the screen and the control from the mouse and keyboard are transferred to the browser.
The user can hence manipulate the webpage with their mouse and keyboard. 
A "Close" button has been added to the master widget (the widget that contains the web browser) so that
the player can close it to return to the game.
Since the web browser API is very limited, the points the player scored in the challenge, 
could not be transferred easily to the game. Hence, a workaround had to be found.
This is what the next paragraph will explore.

Two different tools were used to transfer the points between the game and the challenges.
VaREST is a plugin for Unreal Engine 4 which allows the developer to make RESTful calls through Unreal Engine 4 blueprints.
Even though the developer can program with C++ in Unreal Engine 4, 
Blueprints \footnote{https://docs.unrealengine.com/en-US/Engine/Blueprints/index.html} 
provide an easy programming interface so that the developer does not have to write C++ code.
VaREST allows the developer to make API calls and parse JSON objects through the Blueprints. 
Therefore, the developer does not have to install Visual Studio to make these calls in C++.
The second tool that was used to solve the problem is the Express.js backend.
Express.js \footnote{https://expressjs.com/} is Node.js web framework which allows the developer to build an API.
The Express.js backend had three routes: one for each challenge. Each challenge would have a GET and a POST route.
The POST route is used by the web application to pass the points. 
The GET route is used by the open-world game to fetch the points.
Now that these two tools have been clarified, the implementation will be presented.
Firstly, the web application was deployed on Amazon Web Services (AWS) so that a url could be obtained.
Then, the initial URLs were assigned to the three web browser widgets (one for each challenge).
Afterwards, the backend was implemented and also deployed on AWS.
That done, the web application would send the points in a JSON object to the backend each time the player would complete a challenge.
Finally, when the user would press the "Close" button in the widget in the game, 
this would trigger a request to the backend, the points would be fetched, and added to the overall points in the game.
Figure \ref{fig:pointsImp} presents a visualisation of how this system works.

\begin{figure}[b]
    \centering
    \includegraphics[width=\textwidth, frame]{images/PointsImplementation.pdf}
    \caption{This chart presents how points are transferred from the web application to the open-world game.}
    \label{fig:pointsImp}
\end{figure}

\section{Conclusion}

This chapter presented the implementation of the project where the three implementation steps were presented. 
Firstly, the web application containing the challenges was implemented.
Then, the objects were placed on the map in locations where the users was to interact with them.
Finally, everything was connected together using a backend and RESTful calls.
The next chapter will present the planned evaluation of this implementation.

%==================================================================================================================================
\chapter{Evaluation} 

This chapter will present the planned evaluation for the implementation presented above.
The evaluation was planned to be face-to-face as the whole game was planned to be run on one computer.
This is to avoid wasting time by asking the subjects to install the game on their computers.
Each subject was to be assigned a unique number to keep their responses anonymous.
The main aim of this evaluation is to verify that the goals set out in the requirements chapter have been completed successfully.
Also, another goal is to check how well the gamification characteristics identified by \citet{cugelman_gamification:_2013} were designed 
and implemented.
In this perspective, three different evaluation strategies were used.
The first one was two evaluation forms that would ask the subjects questions about cryptography and concepts presented in the game.
The first form would have been presented before the subject plays the game, and the second one after they play the game.
Secondly, one form has been used to evaluate the gamification characteristics for each challenge identified by \citet{cugelman_gamification:_2013}.
Finally, the web application was to collect some data concerning the challenges.
Each of these evaluation methods will be developed in the following sections.
Finally, due to the 2020 Coronavirus pandemic, the evaluation could not have been run as it involved face-to-face meetings which 
were prohibited by the University of Glasgow.

\section{Learning evaluation}

To design the learning evaluation, research was done to find methodologies used to evaluate learning progression.
\citet{akpolat_enhancing_2014}'s work was presented in the background chapter of this paper. 
They developed a game where extreme software engineering practices (pair programming, for example) were encouraged.
Students were split up into teams and each week they had to adopt a new practice.
To check whether a group understood the practices correctly, they would pick one person out of the team 
to evaluate the understanding of the team about the practice of the week.
Also, \citet{akpolat_enhancing_2014} ran two surveys: the former after midterm, and the latter after the end of term.
At both occasion, they asked students to evaluate their learning experience and their coding performance.
These ideas have been incorporated into the evaluation of this project.
\citet{su_mobile_2015} implemented a mobile game to teach fourth graders (10-11 years old) about insects.
To evaluate their learning experience, they ran two surveys. 
The first survey was run before they play the mobile game to gather information about what the students know about insects.
The second survey was run after they play the mobile game to gather information about what that students have learned after four weeks.
For both tests, they computed their scores and ran hypothesis tests on it.
Thanks to these tests, they concluded that learning has occurred.

Using the methodologies presented above, the questionnaires were designed.
To evaluate learning, two separate surveys were used.
The first survey was to be taken before the subject play the game.
It asked ten questions concerning material present in all challenges and all of them were multiple choice questions 
except the last one which was "select all that apply".
The first two questions concerned the first challenge. 
These questions were asked: "What is a stream cipher?" and "In what of the following groups does the stream cipher belong?".
The subjects had to choose the correct among the possible choices.
The next seven questions were about the second challenge.
In this set of questions, the subject is asked about the following material:
\begin{itemize}
    \item Purpose of each type cryptographic algorithm (symmetric, asymmetric, and hashing);
    \item What key should be used with what algorithm;
    \item Specific algorithms and their vulnerabilities ("Among these algorithms, which one is outdated?", for example).
\end{itemize}
Finally, the last question targeted the last challenge.
For this question, the subject was to choose among all the choices he is given, the ones that are a serious vulnerability.
Among the possible choices were for example: writing your own algorithm or keeping the plaintext that has just been encrypted.
At the end of this survey, a total score was to be computed.

The first part of the second survey asked the same questions as in the first survey.
The purpose is to compute a total score and then run a hypothesis test to determine whether learning has occurred.
This idea has been implemented by \citet{su_mobile_2015}.
The second part of the survey asked questions about learning quality 
and opinions about student's feelings concerning the game being a complement to the CSF course.
For both of these evaluation goals, a mix of Likert scale, qualitative, and binary (yes or no) questions were used.
For Likert scale questions, the subjects were to choose a value between one and ten.
For qualitative questions, the subjects were to be asked open-ended questions 
where they would have full control over their answer.
For learning quality, the subjects were to be asked to rate the improvement of their knowledge in cryptography
on a scale from 1 to 10. Then, they were to be asked to write what they knew about cryptography before playing the game, 
and what they know after playing it. This is to see whether students believe that playing this game will improve their skills.
The next questions concerned the game complementing the CSF course.
These set of questions aim to evaluate whether the main goal set out in the Requirements chapter was completed successfully. 
The subjects were asked to rate on a scale from 1 to 10 whether they think the game is a good supplement for the CSF course 
and to explain their rating.
Then, they were asked whether their understanding of cryptography improved regarding to the material presented in lectures, 
to specify what they understand after playing it, and they still do not understand.
The second-to-last question asked if the game created any confusion concerning the material, 
and if it did to specify what caused it.
Finally, the last question asked about what the subjects wished was covered in the game.
This is to see what future work could be done to extend the game.

The two forms presented in this section were meant to evaluate the Intended Learning Outcomes identified in the Requirements chapter.
A two-step evaluation was implemented so that a hypothesis test could be run.
The null hypothesis is: the game does not improve the learning in cryptography in students taking CSF.
The alternative hypothesis is: the game does not improve the learning in cryptography in students taking CSF.
Unfortunately, due to the lack of results, the test was not run.
For more details, the two evaluation forms are included in the appendices of this paper.

\section{Gamification Characteristics Evaluation}

This section presents the evaluation form of the gamification characteristics identified by \citet{cugelman_gamification:_2013}.
These provide a good framework to check whether the goals were clearly defined, the challenges were not too difficult,
the feedback of the challenges was clear, the challenges motivated players to continue playing, 
the player were motivated to communicate about the challenges, and whether the challenges were fun to play.
For each of the three challenges, the same questions were asked.
The form has been included in the appendices of this paper.

The first question asked whether the goals were clearly established. 
The subjects had to rate it on a scale from one through ten, one being not clear at all, and ten being very clear.
This allows to check how clear the goals were described to the players.
The second question was a modified Likert scale question: 
the subjects were asked whether the challenge was too easy, too difficult, or just alright.
Feedback from this question gives perspective about how difficult the challenges were to the students,
and if they should be redesigned to make them more accessible or more complicated.
Then, subjects were asked whether the feedback from the challenges was clear. 
If it is not, it would a good idea to think about why it is not, and redesign it.
Afterwards, subjects were asked if they felt motivated to continue playing the game.
This question not only evaluates the challenge, but also the whole open-world game quest system since 
points and unlocking new skills are supposed to motivate the player to keep going.
The next two questions evaluated the interaction between subjects.
The first question asked whether subjects talked with each other about the challenges, and, if they did, 
to specify what they talked about. This is to see how the game influences cooperation and exchange of knowledge.
The second question asked whether they compared their performance. 
This enables to check whether there is competition between subjects to outperform each other, 
and hence more motivation to play the game.
The final question of this survey was about how fun the game was to play for the subjects.
They had to rate this question on a scale from one to ten. 
This is also an important factor in motivation: if the game is not fun, students will not bother playing it.

\section{Challenge Data}

As the web application was connected with a backend, 
this allowed to store some information about the way the user played the challenges.
To accomplish this, the backend was hooked up with a mongodb database.
MongoDB \footnote{https://www.mongodb.com/} is a no-SQL database system which stores JSON documents.
This is convenient as the web application transfers data in JSON objects to the backend.
For each of the challenges, the following elements were recorded:
\begin{itemize}
    \item Time it took the subject to complete the challenge;
    \item The answers the subject tried while completing the challenges;
    \item The number of attempts;
    \item The points the subject obtained.
\end{itemize}
The time allows to see if the challenges were difficult. If the subjects spent too much time on the challenges,
then it might be a good idea to make them easier.
Analysing the answers and the number of attempts allows to check 
whether subjects are trying to brute force their way through the challenges and to check the difficulty.
Finally, the number of points allows to see how well the subjects did in a challenge.

\section{Conclusion}

This section presented the evaluation techniques that were to be used to the evaluate the project.
Firstly, the learning experience evaluation was discussed. 
The aim of this part was to establish if the intended learning outcomes were met.
Then, the gamification characteristics evaluation was described.
The goal was to check whether the game was enjoyable to the player.
Finally, the challenge data evaluation was presented.
Its purpose was to supplement the information collected in the first and second parts of the whole evaluation.
Unfortunately, due to the 2020 Coronavirus pandemic, the evaluation has not been run.
The next and final chapter will summarise and conclude this paper.

%==================================================================================================================================
\chapter{Conclusion}

\section{Project summary}

The main goal of this project was to incorporate a cryptography quest

\section{Future work}

Summarise the whole project for a lazy reader who didn't read the rest (e.g. a prize-awarding committee).
\section{Guidance}
\begin{itemize}
    \item
        Summarise briefly and fairly.
    \item
        You should be addressing the general problem you introduced in the
        Introduction.        
    \item
        Include summary of concrete results (``the new compiler ran 2x
        faster'')
    \item
        Indicate what future work could be done, but remember: \textbf{you
        won't get credit for things you haven't done}.
\end{itemize}

%==================================================================================================================================
%
% 
%==================================================================================================================================
%  APPENDICES  

\begin{appendices}

\chapter{Appendices}

Typical inclusions in the appendices are:

\begin{itemize}
\item
  Copies of ethics approvals (required if obtained)
\item
  Copies of questionnaires etc. used to gather data from subjects.
\item
  Extensive tables or figures that are too bulky to fit in the main body of
  the report, particularly ones that are repetitive and summarised in the body.

\item Outline of the source code (e.g. directory structure), or other architecture documentation like class diagrams.

\item User manuals, and any guides to starting/running the software.

\end{itemize}

\textbf{Don't include your source code in the appendices}. It will be
submitted separately.

\end{appendices}

%==================================================================================================================================
%   BIBLIOGRAPHY   

% The bibliography style is abbrvnat
% The bibliography always appears last, after the appendices.

\bibliographystyle{abbrvnat}

\bibliography{l4proj}

\end{document}
